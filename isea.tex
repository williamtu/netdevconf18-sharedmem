% This file is isea.tex.  It contains the formatting instructions for and acts as a template for submissions to ISEA 2015.  It is based on the ICCC  formats and instructions.  It uses the files isea.sty, isea.bst and isea.bib, the first two of which also borrow from AAAI IJCAI formats and instructions.
% Modified from ICCC.tex by B. Bogart

\documentclass[letterpaper]{article}
\usepackage{isea}
\usepackage[pdftex]{graphicx}
\usepackage{times}
\usepackage{helvet}
\usepackage{courier}
\usepackage[numbers]{natbib}
\pdfinfo{
/Title (Shared Memory Pool for Representors)
/Author (Netdev Conference 0x18, 2024)}
% The file isea.sty is the style file for ISEA 2015 proceedings.
% reference: epoll paper: Busy Polling: Past, Present, Future
% https://netdevconf.info/2.1/papers/BusyPollingNextGen.pdf
% reference: ./Documentation/networking/representors.rst
% old slides about eswitch
% https://www.netdevconf.info/1.2/slides/oct6/04_gerlitz_efraim_introduction_to_switchdev_sriov_offloads.pdf

\title{Shared Memory Pool for Representors}
\author{William Tu, Michal Swiatkowski, and Yossi Kuperman\\
Nvidia and Intel\\
witu@nvidia.com, michal.swiatkowski@intel.com, yossiku@nvidia.com\\
\newline
\newline
}
\setcounter{secnumdepth}{0}

% due deligent https://lore.kernel.org/netdev/39dbf7f6-76e0-4319-97d8-24b54e788435@nvidia.com/
%
\begin{document} 
\maketitle
\begin{abstract}

\end{abstract}

\section{Introduction}
Modern network devices support advanced switching and offloading
capabilities, called switchdev mode.
In switchdev mode, users can create multiple virutal ports, vports,
through sysfs and depending on their use cases, assigned to virtual
machine or containers, or use it as another network device.
Today, virtual ports in switchdev includes PF (primary functions of
a PCIe device, providing full access to the device's capabilities.
PFs can manage multiple VFs and SFs and are responsible for the
overall control and configuration of the device.),
VF (VFs are a type of PCIe function that allows for the partitioning
of a physical network device into multiple virtual devices. ), and SF
SFs, subfunction or scalable function, are portions of a PCI device
that share resources with physical function.

In switchdev mode, a special netdev called representors are created to
provides the slow path for traffic which does not hit any offloaded
fast-path rules in the virtual switch.

representor also 
configure the network connection the representee sees, e.g.
 link up/down, MTU, e
In typical use case, representor and representee are 1:1, e.g,

%are slow-path ports that handle the
%miss traffic, i.e., traffic not being forwarded by the hardware.
As Representor ports are regular Ethernet devices, with multiple channels
consuming DMA memory. Memory consumption of the representor
port's rx buffers can grow to several GB when scaling to 1k VFs reps.
For example, in mlx5 driver, each RQ, with a typical 1K descriptors,
consumes 3MB of DMA memory for packet buffer in descriptors, and with
four channels, it consumes 4 * 3MB * 1024 = 12GB of memory. Since rep
ports are for slow path traffic, most of these rep ports' rx DMA memory
is idle when flows are forwarded directly in hardware to VFs.
In the recent case of smartnic or DPU (data processing unit) such as
Nvidia BlueField and Intel IPU, only the VFs and SFs and a light-weight PF
is exposed to the host side, the PF and representors are created in the
embedded CPU of the smartNIC.
The SmartNIC usually has less memory resources, Nvidia BlueField 3 has
32GB and Intel IPU has NGB, which makes the memory resource more precious.

In this talk we address the memory utilization challenges in DPU, by
first going through all existing solutions in different vendor drivers,
intel, nvidia, broadcom, netronme, and sfc. And we present the shared
memory pool design and implementation.
We should that by sharing the memory 

\section{Background}
%challenges and takeaways
\begin{figure}[h]
\includegraphics[width=3.31in]{figure.png}
\caption{This is an example of figure caption. Note that all figures, and tables are to be referenced in the text. \copyright Respect Copyright.}
\end{figure}

\subsection{Eswitch and Switchdev Mode}
An eSwitch, or embedded switch, is a hardware component within modern network
controllers, such as the Intel® Ethernet Controller E810. It is also known as
a Virtual Ethernet Bridge (VEB). The eSwitch is managed by the Physical Function (PF)
driver of the Ethernet or in the case of SmartNIC, managed by the ECPF (Embedded CPU
s PF) that runs in the smartNIC's core. Eswitch can operate in two modes:
Legacy mode and Switchdev mode. 

Legacy mode operates based on traditional MAC/VLAN steering rules. Switching
decisions are made based on MAC addresses, VLANs, etc. There is limited ability
to offload switching rules to hardware.
On the other hand, switchdev mode allows for more advanced offloading
capabilities of the E-Switch to hardware. In switchdev mode, more switching
rules and logic can be offloaded to the hardware switch ASIC. It enables
representor netdevices that represent the slow path of virtual functions (VFs)
or scalable-functions (SFs) of the device. See more information about
% $$:`Documentation/networking/switchdev.rst <switchdev>` and
%:ref:`Documentation/networking/representors.rst <representors>`.
% devlink dev eswitch show
% devlink dev eswitch set mode switchdev
% need to disable OVS hwoffload
% so it's the iperf TX traffic on VF port that's going to overflow the PF's rxq

\subsection{representors}

A representor has three main roles.
1. It is used to configure the network connection the representee sees, e.g.
   link up/down, MTU, etc.  For instance, bringing the representor
administratively UP should cause the representee to see a link up / carrier
   on event.

2. It provides the slow path for traffic which does not hit any offloaded
   fast-path rules in the virtual switch.  Packets transmitted on the
   representor netdevice should be delivered to the representee; packets
  transmitted by the representee which fail to match any switching rule should
  be received on the representor netdevice.  (That is, there is a virtual pipe 
  connecting the representor to the representee, similar in concept to a veth
  pair.)
  This allows software switch implementations (such as OpenVSwitch or a Linux
   bridge) to forward packets between representees and the rest of the network.
   
 3. It acts as a handle by which switching rules (such as TC filters) can refer
 to the representee, allowing these rules to be offloaded.

 
 The combination of 2) and 3) means that the behaviour (apart from performance)
 should be the same whether a TC filter is offloaded or not.  E.g. a TC rule
 on a VF representor applies in software to packets received on that representor
 netdevice, while in hardware offload it would apply to packets transmitted by
  the representee VF.  Conversely, a mirred egress redirect to a VF representor
  corresponds in hardware to delivery directly to the representee VF.
 
When the system boots, and before any offload is configured, all packets from
the virtual functions appear in the networking stack of the PF via the
representors.  The PF can thus always communicate freely with the virtual functions.

The PF can configure standard Linux forwarding between representors, the uplink
or any other netdev (routing, bridging, TC classifiers).

Thus, a representor is both a control plane object (representing the function in
administrative commands) and a data plane object (one end of a virtual pipe).

As a virtual link endpoint, the representor can be configured like any other
netdevice; in some cases (e.g. link state) the representee will follow the
representor's configuration, while in others there are separate APIs to
configure the representee.
 
A "representee" is the object that a representor represents.  So for example in
the case of a VF representor, the representee is the corresponding VF.

In addition, the devlink E-Switch also comes with other attributes listed
in the following section.

\subsection{Representor queues and its depth}
A network device driver consists of several channels and each channel/queue
represents an NAPI context and a set of queues that trigger that IRQ.
devlink-sd considers only the *regular* RX queue in each channel,
e.g., mlx5's non-regular RQs such as XSK RQ and drop RQ are not applicable
here. Each device driver receives packets by setting up RQ, and
each RQ receives packets by pre-allocating a dedicated set of rx
ring descriptors, with each descriptor pointing to a memory buffer.
The ``shared descriptor pool`` is a descriptor and buffer sharing
mechanism. It allows multiple RQs to use the rx ring descriptors
from the shared descriptor pool. In other words, the RQ no longer has
its own dedicated rx ring descriptors, which might be idle when there
is no traffic, but it consumes the descriptors from the descriptor
pool only when packets arrive.

The shared descriptor pool contains rx descriptors and its memory
buffers. When multiple representors' RQs share the same pool of rx
descriptors, they share the same set of memory buffers. As a result,
the heavy-traffic representors can use all descriptors from the pool,
while the idle, no-traffic representor consumes no memory. All the
descriptors in the descriptor pool can be used by all the RQs. This
makes the descriptor memory usage more efficient.


% explain eswitch, virtual port vport, representors, and port
% eswitch has vport
% vport nic 
As software-defined network and standards such as OpenFlow or tc-flower
requires more and more performance, network card vendors started to offer
complex virtualization capabilities than the legacy layer 2 mac/vlan switch
SR-IOV features, this leads to many new features and design in hardware offload
and device abstraction in Linux kernel.
This section talks about switchdev mode, the mode that enables
modern advanced hardware offloading capabilities as a hardware switch device,
and its data plance port function, vport or virtual port, and the vport's
control plane object, called representors.

\section{Design}
table here shows existing driver's implementation

\subsection{Nvidia mlx5 driver}
\subsection{Intel ice driver}
\subsection{Devlink Interface}

\section{Challenges}
\section{Thanks}

\subsection{Length of Papers}
A variety of paper lengths will be accepted under different categories. Please note that submission lengths must be all inclusive (including references, biographies and acknowledgements).
\begin{itemize}
\item Long paper submissions can be up to 8 pages in the ISEA2015 template format.
\item Art and Research short paper submissions can be up to 4 pages in the ISEA2015 template format.
\item Extended abstract submissions can be up to 2 pages in the ISEA2015 template format.
\item Round tables and square panel submissions can be up to 2 pages in the ISEA2015 template format.
\item Institutional presentation submissions can be up to 2 pages in the ISEA2015 template format.
\end{itemize}

\section{Style and Format}
Templates that implement these instructions can be retrieved electronically at {\small \tt http://isea2015.org}

\subsection{Layout}

Print manuscripts two columns to a page, in the manner in which these instructions are printed. The exact dimensions for pages are:
\begin{itemize}
\item left and right margins: 0.75''
\item column width: 3.31''
\item gap between columns: 0.38''
\item top margin—first page: 1.25''
\item top margin—other pages: 0.75''
\item bottom margin: 1.25''
\end{itemize}

\subsection{Format of Electronic Manuscript}

For the production of the electronic manuscript, you must use {\em Adobe's Portable Document Format} (PDF). Additionally, you must specify the American {\em letter} format (corresponding to 8-1/2'' x 11'') when formatting the paper.

\subsection{Blind Review}

All papers will be reviewed in a single blind manner.  You are at liberty to include your affiliation and cite your papers in a natural manner, and you are also at liberty to anonymize the text if you so desire, in which case, keeping your identity secret is your responsibility.

\subsection{Title and Author Information}

Center the title on the entire width of the page in a 16-point bold font. Below it, center the author name(s) in a 12-point bold font, and then center the address(es) in a 9-point regular font. Credit to a sponsoring agency can appear in the Acknowledgment Section described below.

\begin{figure}[h]
\includegraphics[width=3.31in]{figure.png}
\caption{This is an example of figure caption. Note that all figures, and tables are to be referenced in the text. \copyright Respect Copyright.}
\end{figure}

\subsection{Abstract}

The title ``Abstract'' should be 10 point, bold type, centered at the beginning of the left column. The body of the abstract summarizing the thesis and conclusion of the paper in no more than 200 words should be 9 point, justified, regular type.

\subsection{Text}

The main body of the text immediately follows the abstract. Use 10-point type in {\em Times New Roman} font.

Indent when starting a new paragraph, except after major headings. 

\subsection{Headings and Sections}

When necessary, headings should be used to separate major sections of your paper. (These instructions use many headings to demonstrate their appearance; your paper should have fewer headings.

\subsubsection{Section Headings}

Print section headings centered, in 12-point bold type in the style shown in these instructions. Your body text should be 10 point, justified, single space. Do not number sections.

\subsubsection{Subsection Headings}

Print subsection headings left justified, in 11-point bold type and mixed case (nouns, pronouns, and verbs are capitalized). They should be flush left. Your text should be 10 point, justified, single space. Do not number subsections.

\subsubsection{Subsubsection Headings}

Print subsubsection headings inline in 10-point bold type. Do not number subsubsections.

\subsubsection{Special Sections}

You may include an unnumbered acknowledgments section, including acknowledgments of help from colleagues, financial support, and permission to publish.

The references section is headed ``References,'' printed in the same style as a section heading. A sample list of references is given at the end of these instructions.  Note the various examples for books, proceedings, multiple authors, etc. 

\subsection{Footnotes}

If footnotes are necessary, place them at the bottom of the page in 9-point font. Refer to them with superscript numbers.\footnote{This is how your footnotes should appear.} Separate them from the text by a short horizontal line. 

\subsection{Itemized Lists}

Itemized lists shall use the en-dash as item. Let’s take the case of URL, automatic links and punctuation as an example:
Turn off the automatic linking feature for URLs in Word.

Quotations: For direct quotations remember to use ``double inverted commas.'' Quotations must be carefully transcribed and accurate. 

Periods and commas go inside quotation marks. This applies to ``double inverted commas,'' as well as single `inverted commas,' and to the use of a full stop as in the ``following example.'' 

Parenthesis: When an entire sentence is enclosed in parentheses, the punctuation mark belongs inside the closing parenthesis as in this example: applying this may be difficult at times. (We think it is important.)

\begin{itemize}
\item The punctuation mark belongs outside the closing parenthesis if the brackets are within the sentence as in this example: applying this may be difficult at times, but good results are guaranteed (and this is important).
\item Use en dashes with spaces -- like this -- to set off phrases. En dashes are moreover placed between digits to indicate a range (1--10 October; pp. 25--30). You can type an en dash with ALT + 0150 (in the numeric keypad) in Windows, or OPTION + HYPHEN in Mac.
\end{itemize}

\subsection{Quotations and Extracts}
Indent long quotations and extracts by 10 points at left margins.

\section{Acknowledgments}
The preparation of these instructions and the \LaTeX{} and Word files was facilitated by borrowing from similar documents used for ICCC proceedings.

\begin{figure*}
\includegraphics[width=\textwidth]{two-column-figure.png}
\caption{Example of a double-column figure with caption. \copyright Respect Copyright.}
\end{figure*}

\section{Bibliography}
The title ``Bibliography'' should be 12 point, bold style, centered. Using 9 point, regular type, list your bibliography in alphabetical order by family name, after the references. The difference between a reference list and a bibliography is that in your references, you list all the sources you directly referred to in the body of your writing in numerical order, whereas a bibliography includes an alphabetical listing of all those authors and sources that you have consulted while writing your essay. Use the same format as for the references otherwise. Those using Latex will follow the usual cite command format \cite{boden92}.

\section{Author(s) Biography(ies)}
The title ``Author(s) Biography(ies)'' should be 12 point, bold style, centered. Using 9 point, regular type, biographies should be no longer than 150-word count.

\section{Questions?}

For technical questions about Microsoft Word formatting please seek online tutorials. For other questions about your manuscript please contact: {\tt ISEA2015-info@sfu.ca}


\bibliographystyle{isea}
\bibliography{isea}


\end{document}
